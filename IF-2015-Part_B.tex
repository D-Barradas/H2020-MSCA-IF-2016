\documentclass[a4paper,11pt]{article}

\usepackage[T1]{fontenc}
\usepackage[utf8]{inputenc}
\usepackage{lmodern}
\usepackage{eurosym}
\usepackage{lastpage}
\usepackage{xspace}
\usepackage[margin=16mm,includehead,includefoot]{geometry}
\usepackage{fancyhdr}
\usepackage{booktabs}
\usepackage{graphicx}
\usepackage{multirow}
\usepackage{array}
\usepackage[usenames,dvipsnames]{xcolor}
\usepackage{csquotes}
\usepackage{pgfgantt}
\usepackage[compact]{titlesec}
\usepackage{enumitem}
\usepackage[style=verbose-ibid,backend=bibtex]{biblatex}
\usepackage{hyperref}
\usepackage{amsfonts,amsmath,amssymb}
\usepackage{lineno}
\usepackage[all]{hypcap}  %for hyperlinks to point to the top of the figures and tables

\newcommand{\TODO}[1]{{\textcolor{red}{[\textbf{TODO:} #1]}}}
\newcommand{\acronym}{{\sc Proposal Acronym}\xspace}
% \titlespacing\section{0pt}{6pt plus 4pt minus 2pt}{0pt plus 2pt minus 2pt}
% \titlespacing\subsection{0pt}{6pt plus 4pt minus 2pt}{0pt plus 2pt minus 2pt}
% \titlespacing\subsubsection{0pt}{6pt plus 4pt minus 2pt}{0pt plus 2pt minus 2pt}

\titleformat*{\section}{\large\bfseries}
\titleformat*{\subsection}{\normalsize\bfseries}
\titleformat*{\subsubsection}{\normalsize\bfseries}
\let\oldfootnotesize\footnotesize
\renewcommand{\footnotesize}{\fontsize{8bp}{1em}\selectfont}
\renewcommand{\cite}{\autocite} % citations in footnotes
\bibliography{bibliography}

\headheight=14pt

\hypersetup{
    pdftitle={H2020-MSCA-IF-2015},    % title
    pdfauthor={},
    colorlinks=true,
    citecolor=black,
    linkcolor=black,
    urlcolor=blue
  }

\pagestyle{fancy}
\fancyhead{}
\fancyhead[C]{\acronym\xspace - Standard EF / CAR / RI / GF}
\fancyfoot{}
\fancyfoot[C]{Part B - Page \thepage~of \pageref{LastPage}}


\renewcommand{\headrulewidth}{0pt}


\renewcommand{\contentsname}{TABLE OF CONTENTS}


\begin{document}

\phantom{a}
\vspace{15mm}
\begin{center}


        \Large{
      
     
        \textbf{START PAGE}
  
          \vspace{15mm}
          MARIE SKŁODOWSKA-CURIE ACTIONS\\
          \vspace{1cm}
          
          \textbf{Individual Fellowships (IF)}\\
          \textbf{Call: H2020-MSCA-IF-2015}
          \vspace{2cm}                   

          PART B
          \vspace{2.5cm}

          ``\acronym''
          \vspace{2cm}

          \textbf{This proposal is to be evaluated as:}
          \vspace{.5cm}

          \textbf{[Standard EF] [CAR] [RI] [GF]}\\
        }
        \large{[Delete as appropriate]}

  \end{center}
\vspace{1cm}

\newpage
\setcounter{tocdepth}{1}
\setcounter{section}{-1}
\tableofcontents


\newpage
\section{LIST OF PARTICIPANTS}
\label{sec:participants}

\newcommand\rotx[1]{\rotatebox[origin=c]{90}{\textbf{#1}}}
\newcommand\roty[1]{\rotatebox[origin=c]{90}{\parbox{4cm}{\raggedright\textbf{#1}}}}
\newcommand\MyHead[2]{\multicolumn{1}{l|}{\parbox{#1}{\centering #2}}}

\noindent\begin{tabular}{|m{2.4cm}|m{1cm}|b{1em}|b{1em}|c|m{2.5cm}|m{2cm}|c|}
\hline
  \textbf{Participants}
& \MyHead{1cm}{\textbf{Legal\\Entity\\Short\\Name}}
& \rotx{Academic}
& \rotx{Non-academic}
& \textbf{Country}
& \MyHead{2.1cm}{\textbf{Dept. / \\Division / \\Laboratory}}
& \textbf{Supervisor}
& \MyHead{2.5cm}{\textbf{Role of\\Partner\\Organisation}} \\
\hline
\underline{Beneficiary} & & & & & & & \\\hline
- NAME  & & & & & & & \\\hline
\underline{Partner} \underline{Organisation} & & & & & & & \\\hline
- NAME  & & & & & & & \\\hline
\end{tabular}
\vspace{\baselineskip}

Data for non-academic beneficiaries

\noindent\begin{tabular}{|m{1.7cm}|m{2cm}|m{1.8cm}|c|c|m{2.5cm}|c|c|c|}
\hline
  \textbf{Name}
& \roty{Location of research premises (city / country)}
& \roty{Type of R\&D activities}
& \roty{No. of fulltime employees}
& \roty{No. of employees in R\&D}
& \roty{Website}
& \roty{Annual turnover (approx. in Euro)}
& \roty{Enterprise status (Yes/No)}
& \roty{SME status  (Yes/No)}
\\\hline
& & & & & & & & \\\hline
\end{tabular}
\vspace{\baselineskip}

Note that:
\begin{itemize}
\item Any inter-relationship between different participating institutions or individuals (e.g. family ties, shared premises or facilities, joint ownership, financial interest, overlapping staff or directors, etc.) \textbf{must} be declared and justified \textbf{in this part of the proposal};
\item The information in the table for non-academic beneficiaries \textbf{must be based on current data, not projections};
\item The data provided relating to the capacity of the participating institutions will be subject to verification during the Grant Agreement preparation phase.
\end{itemize}


\newpage

\section{EXCELLENCE}
\label{sec:excellence}

Please note that the principles of the European Charter for Researchers and Code of Conduct for the Recruitment of Researchers promoting open recruitment and attractive working conditions are expected to be endorsed and applied by all beneficiaries in the Marie Sklodowska-Curie actions.

\subsection{Quality, innovative aspects and credibility of the research (including inter/multidisciplinary aspects)}
\label{sec:quality}

You should develop your proposal according to the following lines:
\begin{itemize}
\item Introduction, state-of-the-art, objectives and overview of the action
\item Research methodology and approach: highlight the type of research and innovation activities proposed
\item Originality and innovative aspects of the research programme: explain the contribution that the project is expected to make to advancements within the project field. Describe any novel concepts, approaches or methods that will be employed.
\end{itemize}
Explain how the high-quality, novel research is the most likely to open up the best career possibilities for the Experienced Researcher and new collaboration opportunities for the host organisation(s). 

\subsection{Clarity and quality of transfer of knowledge/training for the development of the researcher in light of the research objectives}
\label{sec:transfer}

Outline how a two way transfer of knowledge will occur between the researcher and the host institution, in view of their future development and past experience:
\begin{itemize}
\item Explain how the \emph{Experienced Researcher} will gain new knowledge during the fellowship at the hosting organisation(s)
\item Outline the previously acquired knowledge and skills that the researcher will transfer to the host organisation.
\end{itemize}
For Global Fellowships explain how the newly acquired skills and knowledge in the Third Country will be transferred back to the host institution in Europe during the incoming phase.

\subsection{Quality of the supervision and the hosting arrangements}
\label{sec:supervision}

Required sub-headings:

\subsubsection*{Qualifications and experience of the supervisor(s)}

Information regarding the supervisor(s) must include the level of experience on the research topic proposed and document their track record of work, including main international collaborations. Information provided should include participation in projects, publications, patents and any other relevant results.

\subsubsection*{Hosting arrangements}

The text must show that the Experienced Researcher should be well integrated within the hosting organisation(s) in order that all parties gain the maximum knowledge and skills from the fellowship. The nature and the quality of the research group/environment as a whole should be outlined, together with the measures taken to integrate the researcher in the different areas of expertise, disciplines, and international networking opportunities that the host could offer.

For GF both phases should be described - for the outgoing phase, specify the practical arrangements in place to host a researcher coming from another country, and for the incoming phase specify the measures planned for the successful (re-)integration of the researcher.

Describe briefly how the host will contribute to the advancement of their career.  In that context the following section of the European Charter for Researchers refers specifically to career development:

\paragraph{Career development}
Employers and/or funders of researchers should draw up, preferably within the framework of their human resources management, a specific career development strategy for researchers at all stages of their career, regardless of their contractual situation, including researchers on fixed-term contracts. It should include the availability of mentors involved in providing support and guidance for the personal and professional development of researchers, thus motivating them and contributing to reducing any insecurity in their professional future. All researchers should be made familiar with such provisions and arrangements. 

Therefore a Career Development Plan should not be included in the proposal, but it is part of implementing the project in line with the European Charter for Researchers.

\subsection{Capacity of the researcher to reach and re-enforce a position of professional maturity in research}
\label{sec:maturity}

Applicants should demonstrate how their proposed research and personal experience can contribute to their professional development as an independent/mature researcher.

Please keep in mind that the fellowships will be awarded to the most talented researchers as shown by the proposed research and their track record (Curriculum Vitae, section~\ref{sec:cv}), in relation to their level of experience.

\section{IMPACT}
\label{sec:impact}

\subsection{Enhancing research- and innovation-related human resources, skills, and working conditions to realise the potential of individuals and to provide new career perspectives}
\label{sec:enhancement}

Explain the expected impact of the planned research and training, and new competences acquired during the fellowship on the capacity to increase career prospects for the Experienced Researcher after this fellowship finishes.
Demonstrate also to what extent competences acquired during the fellowship, including any secondments will increase the impact of the researcher’s future activity on European society, including the science base and/or the economy.  

\subsection{Effectiveness of the proposed measures for communication and results dissemination}

The new knowledge generated by the action should be used wherever possible to
advance research, to foster innovation, and to promote the research profession
to the public. Therefore develop following three points.
\begin{itemize}
\item Communication and public engagement strategy of the action 
\item Dissemination of the research results 
\item Exploitation of results and intellectual property rights
\end{itemize}
Concrete plans for the above must be included in the Gantt Chart (see point~\ref{subsec:work_plan}).

The following sections of the European Charter for Researchers refer specifically to public engagement and dissemination:

\paragraph{Public engagement}
Researchers should ensure that their research activities are made known to society at large in such a way that they can be understood by non-specialists, thereby improving the public's understanding of science. Direct engagement with the public will help researchers to better understand public interest in priorities for science and technology and also the public's concerns. 

\paragraph{Dissemination, exploitation of results}
All researchers should ensure, in compliance with their contractual arrangements, that the results of their research are disseminated and exploited, e.g. communicated, transferred into other research settings or, if appropriate, commercialised. Senior researchers, in particular, are expected to take a lead in ensuring that research is fruitful and that results are either exploited commercially or made accessible to the public (or both) whenever the opportunity arises. 


\section{IMPLEMENTATION}
\label{sec:implementation}

\subsection{Overall coherence and effectiveness of the work plan (including appropriateness of the allocation of tasks and resources)}
\label{subsec:work_plan}

Describe the different work packages. The proposal should be designed in such a way to achieve the desired impact. A Gantt Chart should be included in the text listing the following:
\begin{itemize}
\item Work Packages titles (for EF there should be at least 1 WP); 
\item List of major deliverables;    
\item List of major milestones;  
\item Secondments if applicable.
\end{itemize}

The schedule should be in terms of number of months elapsed from the start of the project.

\subsection{Appropriateness of the management structure and procedures, including quality management and risk management}

Develop your proposal according to the following lines:
\begin{itemize}
\item Project organisation and management structure, including the financial management strategy, as well as the progress monitoring mechanisms put in place;
\item Risks that might endanger reaching project objectives and the contingency plans to be put in place should risk occur.
\end{itemize}


\begin{figure}[!htbp]
\begin{center}

\begin{ganttchart}[
    canvas/.append style={fill=none, draw=black!5, line width=.75pt},
    hgrid style/.style={draw=black!5, line width=.75pt},
    vgrid={*1{draw=black!5, line width=.75pt}},
    title/.style={draw=none, fill=none},
    title label font=\bfseries\footnotesize,
    title label node/.append style={below=7pt},
    include title in canvas=false,
    bar label font=\small\color{black!70},
    bar label node/.append style={left=2cm},
    bar/.append style={draw=none, fill=black!63},
    bar progress label font=\footnotesize\color{black!70},
    group left shift=0,
    group right shift=0,
    group height=.5,
    group peaks tip position=0,
    group label node/.append style={left=.6cm},
    group progress label font=\bfseries\small
  ]{1}{24}
  \gantttitle[
    title label node/.append style={below left=7pt and -3pt}
  ]{Month:\quad1}{1}
  \gantttitlelist{2,...,24}{1} \\
  \ganttgroup{Work Package}{1}{10} \\
  \ganttgroup{Deliverable}{5}{15} \\
  \ganttgroup{Milestone}{5}{5} \\
  \ganttgroup{Secondment}{20}{23} \\
  \ganttgroup{Conference}{16}{16} \\
  \ganttgroup{Workshop}{17}{17} \\
  \ganttgroup{Seminar}{18}{18} \\
  \ganttgroup{Dissemination}{23}{24} \\
  \ganttgroup{Public engagement}{4}{5} \\
  \ganttgroup{Other}{7}{10}
\end{ganttchart}

\end{center}
\caption{Example Gantt Chart}
\end{figure}

\subsection{Appropriateness of the institutional environment (infrastructure)}
\label{sec:institution}

\begin{itemize}
\item Give a description of the main tasks and commitments of the beneficiary and partners (if applicable). 
\item Describe the infrastructure, logistics, facilities offered in as far they are necessary for the good implementation of the action.  
\end{itemize}

\subsection{Competences, experience and complementarity of the participating organisations and institutional commitment}
\label{sec:competences}

The active contribution of the beneficiary to the research and training activities should be described. For GF also the role of partner organisations in Third Countries for the outgoing phase should appear. Additionally a letter of commitment shall also be provided (included within the PDF file of part B, but outside the page limit) for the partner organisations in Third Countries.\newline

\noindent NB: Each participant is described in Section~\ref{sec:capacities}. This specific information should not be repeated here.


\newpage
\section{CV OF THE EXPERIENCED RESEARCHER}
\label{sec:cv}

The CV is intrinsic to the evaluation of the whole proposal and is assessed throughout the 3 evaluation criteria.

This section should be limited to maximum 5 pages and should include \textbf{the standard academic and research record}. Any research career gaps and/or unconventional paths should be clearly explained so that this can be fairly assessed by the independent evaluators.

The \emph{Experienced Researchers} must provide a list of achievements reflecting their track record, and this may include, if applicable:

\begin{enumerate}
\item \textbf{Publications} in \textbf{major, peer-reviewed conference proceedings and/or monographs} of their respective research fields, indicating also the number of citations (excluding self-citations) they have attracted.
\item Granted \textbf{patent(s)}.
\item \textbf{Research monographs, chapters} in collective volumes and any translations thereof.
\item \textbf{Invited presentations} to peer-reviewed, internationally established conferences and/or international advanced schools.
\item \textbf{Research expeditions} that the Experienced Researcher has led. 
\item \textbf{Organisation of International conferences} in the field of the applicant (membership in the steering and/or programme committee).
\item Examples of \textbf{leadership in industrial innovation}.
\item \textbf{Prizes and Awards}.
\item Funding received so far.
\item Supervising, mentoring activities.
\end{enumerate}


\newpage
\section{CAPACITIES OF THE PARTICIPATING ORGANISATIONS}
\label{sec:capacities}

All organisations (whether beneficiary or partner organisation) must complete the appropriate table below, which will give input on the profile of the organisation as a whole. Complete one table of maximum one page per institution, beneficiary or partner organisation (min font size: 9). The experts will be instructed to disregard content above this limit.
\vspace{\baselineskip}

{\fontsize{9bp}{1em}\selectfont % should be 9pt
\noindent\begin{tabular}{>{\raggedright}p{.25\textwidth}p{.7\textwidth}}
  \multicolumn{2}{l}{\textbf{Beneficiary X}} \\\midrule
\textbf{General Description} &

\\\midrule
\textbf{Role and Commitment of key persons (supervisor)} &
(names, title, qualifications of the supervisor)

\\\midrule
\textbf{Key Research Facilities, Infrastructure and Equipment} &
Demonstrate that the team has sufficient facilities and infrastructure to host and/or offer a suitable environment for training and transfer of knowledge to recruited Experienced Researcher
\\\midrule
\textbf{Independent research premises?} &
Please explain the status of the beneficiary's research facilities – i.e. are they owned by the beneficiary or rented by it? Are its research premises wholly independent from other beneficiaries and/or partner organisations in the consortium?
\\\midrule
\textbf{Previous Involvement in Research and Training Programmes} &
Detail any relevant EU, national or international research and training projects in which the beneficiary has previously participated
\\\midrule
\textbf{Current involvement in Research and Training Programmes} &
Detail the EU and/or national research and training actions in which the partner is currently participating
\\\midrule
\textbf{Relevant Publications and/or research/innovation products} &
(Max 5) Produced by the organisation, not limited to the supervisor within the organisation.
\\\bottomrule
\end{tabular}}
\vspace{\baselineskip}

{\fontsize{9bp}{1em}\selectfont
\noindent\begin{tabular}{>{\raggedright}p{.25\textwidth}p{.7\textwidth}}
  \multicolumn{2}{l}{\textbf{Partner Organisation Y}} \\\midrule
\textbf{General Description} &

\\\midrule
\textbf{Key Persons and Expertise (supervisor)} &

\\\midrule
\textbf{Key Research facilities, infrastructure and equipment} &

\\\midrule
\textbf{Previous and Current Involvement in Research and Training Programmes} &

\\\midrule
\textbf{Relevant Publications and/or research/innovation product} &
(Max 3)
\\\bottomrule
\end{tabular}}


\newpage
\vspace{15mm}
\begin{center}


        \Large{
      
     
        \textbf{ENDPAGE}
  
          \vspace{15mm}
          MARIE SKŁODOWSKA-CURIE ACTIONS\\
          \vspace{1cm}
          
          \textbf{Individual Fellowships (IF)}\\
          \textbf{Call: H2020-MSCA-IF-2015}
          \vspace{2cm}                   

          PART B
          \vspace{2.5cm}

          ``\acronym''
          \vspace{2cm}

          \textbf{This proposal is to be evaluated as:}
          \vspace{.5cm}

          \textbf{[Standard EF] [CAR] [RI] [GF]}\\
        }
        \large{[Delete as appropriate]}

  \end{center}
\vspace{1cm}


\end{document}